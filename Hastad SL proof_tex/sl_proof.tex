\documentclass{article}
\usepackage[utf8]{inputenc}

\usepackage{todonotes}

\usepackage{amsthm}
\usepackage{amssymb}

%macros
%names
\newcommand{\hastad}{H\aa stad}

%classes%
\newcommand{\cnf}{\mathrm{CNF}}
\newcommand{\aco}{\mathrm{AC}^0}
\newcommand{\acco}{\mathrm{ACC}^0}
\newcommand{\tco}{\mathrm{TC}^0}
\newcommand{\ncone}{\mathrm{NC}^1}
\newcommand{\thrand}{\mathrm{LTF} \circ \mathrm{AND}}
\newcommand{\thrxor}{\mathrm{LTF} \circ \oplus}
\newcommand{\thrmod}[1]{\mathrm{LTF} \circ \mathrm{MOD}_{#1}}
\newcommand{\thrthr}{\mathrm{LTF} \circ \mathrm{LTF}}
\newcommand{\majmaj}{\mathrm{MAJ} \circ \mathrm{MAJ}}
\newcommand{\thrmaj}{\mathrm{LTF} \circ \mathrm{MAJ}}
\newcommand{\MOD}{\mathrm{MOD}}

%functions%
\newcommand{\sym}{\mathrm{SYM}}
\newcommand{\thr}{\mathrm{LTF}}
\newcommand{\maj}{\mathrm{MAJ}}
\newcommand{\IN}{\mathrm{IN}}
\newcommand{\AN}{\mathrm{AND}}
\newcommand{\OR}{\mathrm{OR}}
\newcommand{\xor}{\oplus}
\newcommand{\bool}[1]{\cube{#1} \rightarrow \cube{}}
\newcommand{\fbool}[1]{\fcube{#1} \rightarrow \fcube{}}
\newcommand{\MP}{\mathrm{MP}}
\newcommand{\IP}[1]{\mathrm{IP}_{#1}}
\newcommand{\ghr}{\mathrm{GHR}}
\newcommand{\ut}{\mathrm{UT}}
\newcommand{\sumjsumi}{\sum_{j=0}^{2n-1} \sum_{i=0}^{n-1}}
\newcommand{\sumi}{\sum_{i=0}^{n-1}}
\newcommand{\sumj}{\sum_{j=0}^{2n-1}}

%shortcuts%
\newcommand{\inner}[2]{\left\langle #1,#2 \right\rangle}
\newcommand{\bra}[1]{\left\{#1\right\}}
\newcommand{\setcond}[2]{\left\{#1\: \middle|\: #2\right\}}
\newcommand{\abs}[1]{\left| #1 \right|}
\newcommand{\prob}[2]{\Pr_{#1}\left[#2\right]}
\newcommand{\E}[2]{\mathbb{E}_{#1}\left[#2\right]}
\newcommand{\Ex}[2]{\underset{#1}{\mathbb{E}} \left[ #2 \right]}
\newcommand{\pat}[1]{{#1}^{\mathrm{mask}}}%pattern matrixing
\newcommand{\patt}[2]{{#2}^{\mathrm{op}}_{#1}}
\newcommand{\disc}{\mathrm{disc}}
\newcommand{\cube}[1]{\{0,1\}^{#1}}
\newcommand{\fcube}[1]{\{-1,1\}^{#1}}
\newcommand{\thrdeg}{\mathrm{deg}_{\pm}}
\newcommand{\tsxor}{l_{\oplus}}
\newcommand{\tsand}{l_{\wedge}}
\newcommand{\op}[1]{{#1}^{\mathrm{op}}}
\newcommand{\sbra}[1]{\left[#1\right]} 
\newcommand{\cbra}[1]{\left(#1\right)}

%Symbols%
\newcommand{\IF}{\mathrm{if}}
\newcommand{\F}{\mathbb{F}}
\newcommand{\bigo}[1]{O\left(#1\right)}
\newcommand{\U}{\mathrm{u}}
\newcommand{\p}{\mathcal{P}}
\newcommand{\Z}{\mathbb{Z}}
\newcommand{\R}{\mathbb{R}}
\newcommand{\sgn}[1]{\mathrm{sgn}\left( #1 \right)}
\newcommand{\ie}{i.e.}

%constants
\def\half{\frac{1}{2}}

%communication
\newcommand{\Rpriv}[1]{\mathrm{R}_{#1}}
\newcommand{\Rpub}[1]{\mathrm{R}_{#1}^{\mathrm{Pub}}}
\newcommand{\Dist}[2]{\D^{#1}_{#2}}

%%environments
%\theoremstyle{plain}
\newtheorem{theorem}{Theorem}
%\newtheorem*{theorem*}{Theorem}
\newtheorem{corollary}[theorem]{Corollary}
\newtheorem{lemma}[theorem]{Lemma}
%\newtheorem*{lemma*}{Lemma}
\newtheorem{observation}[theorem]{Observation}
\newtheorem{proposition}[theorem]{Proposition}
\newtheorem{definition}[theorem]{Definition}
\newtheorem{claim}[theorem]{Claim}
%\newtheorem*{claim*}{Claim}
\newtheorem{fact}[theorem]{Fact}
\newtheorem{assumption}[theorem]{Assumption}
\newtheorem{remark}[theorem]{Remark}
\newtheorem{conjecture}{Conjecture}
%\newtheorem*{conjecture*}{Conjecture}
\newtheorem{question}{Question}
%\newtheorem*{question*}{Question}
%\newtheorem*{answer*}{Answer}
%%commands%



\newcommand{\calF}{\mathcal{F}}
\newcommand{\Rp}{\mathcal{R}_p}


\begin{document}

\begin{definition}[Downward closed set of restrictions]
	\label{def: downward closure}
	A set of restrictions $\calF$ is called downward closed if whenever $\rho \in \calF$ 
	and $stars(\rho') \subseteq stars(\rho)$, $\rho' \in \calF$. 
\end{definition}
    
\begin{lemma}
	\label{lem: downclosure intersection}
    Let $\calF$,$\calF'$ be two downward closed subsets of restrictions, then $\calF \cap \calF'$ 
    is also downward closed.  
\end{lemma}

\section{Hastad's proof of SL}

\begin{lemma}[Switching lemma]
	\label{lem: hastad switching lemma}    
    Let $f$ be computed by a depth 2 circuit of bottom fan-in $k$. Let $\calF$ be a downward-closed 
    set of restrictions from $\Rp$. Let $depth(f)$ denote the minimum depth of a decision-tree 
    computing $f$. 
	$$\prob{\rho}{depth(f)>s|\rho \in \calF} \leq (5pk)^s.$$
\end{lemma}
    
\begin{theorem}[Criticality of bounded bottom fan-in CNF/DNF]
	The criticality of  depth 2 circuits with bottom fan-in $k$ is $k$.  
\end{theorem}
    
\begin{proof}[Proof of lemma \ref{lem: hastad switching lemma}]
	Without loss of generality, assume $f$ is a CNF, i.e., $f$ can be written as 
	$$f = \wedge_{i=1}^m C_i$$
	where each $C_i$ is disjunction of at most $t$ literals.The proof is by induction on $m$.        
		        
	The analysis can be divided into two cases depending on whether $\rho$ sets $C_1$ to 1. 
	Let us for the sake of convenience, assume that $C_1$ is a disjunction of the variables 
	$x_1,\ldots ,x_{t_o}$ where $t_0 \leq t$. We can bound the probability of the lemma by the max of 
	following two probabilites
		        
	\begin{equation}
		\label{eq: C1 equiv 1}
		\prob{\rho}{depth(f)>s|\rho \in \calF \wedge C_1|_\rho \equiv 1}
	\end{equation}
	and 
	\begin{equation}
		\label{eq: C_1 not equiv 1}
		\prob{\rho}{depth(f)>s|\rho \in \calF \wedge C_1|_\rho \not\equiv 1}
	\end{equation}
		        
    The first term is taken care of by induction applied to $f$ without its first clause(and thus has 
    at most m-1 cluases).  $\cbra{\rho| C_1|_\rho \equiv 1}$ is a downward closed set and hence by 
    lemma \ref{lem: downclosure intersection}, the conditioning is of the right form. 
		        
    In the second term however, $\cbra{\rho| C_1|_\rho \not\equiv 1}$ is not a downward closed set 
    of restrictions. Therefore bounding in this case needs a bit more work.  
                
    
\end{proof}

\end{document}
